\chapter{Увод}

Рачунари су од свог настанка па до данашњих дана нашли примену у најразличитијим областима живота.
Свака од тих примена придаје различиту важност одређеним карактеристикама рачунара (нпр. перформансе, величина, цена),
што је довело до настанка различитих класа рачунара. Прелазак тржишта са мејнфрејм и персоналних рачунара на данас популарне кластере великих размера и преносиве уређаје, као и експлозија јефтиних уређаја повезаних на мрежу (тзв. интернет ствари), од дизајнера хардвера захтева да преиспитају одлуке и метрике којима су се до сада водили.
Ове нове класе рачунара, уместо на сирове перформансе, акценат стављају на енергетску ефикасност, тј. однос утрошене енергије и количине обрађених података.
Ова промена захтева представља одличну прилику за примену алтернативних архитектура рачунара. Једну од таквих архитектура представљају \textit{\acrfull{risc}} процесори. Иако \textit{\acrshort{risc}} архитектуре нису нова појава (први \textit{\acrshort{risc}} процесори су направљени 1970-их), тренутно тржиште је веома погодно за развој оваквих процесора.

Други фактор који има значајан утицај на развој модерних процесора је смањење брзине којом се повећава број транзистора на једном чипу (тзв. \textit{Moore}-ov закон).
Последица тог смањења је да перформансе и енергетска ефикасност више не могу да се побољшавају повећањем броја транзистора и смањењем њихових димензија, већ побољшања у перформансама примарно долазе из унапређења микро-архитектуре. Што за последицу има јачање монополистичких позиција одређених компанија и тиме негативно утиче на иновативност у том пољу.

Истраживачи на Универзитету Калифорнија, Беркли су увидели ове проблеме те су дизајнирали и објавили \textit{\acrshort{risc}} процесорску архитектуру отвореног кода под именом \рисцв{}\cite{riscv} (изговара се \textit{RISC five}). \рисцв{} архитектура је дизајнирана тако да буде довољно једноставна за примену у едукацији а такође и довољно моћна за примену у истраживачким\cite{rocket} и комерцијалним\cite{sifive}\cite{tenstorrent} пројектима. То је постигнуто модуларним приступом где постоји основна архитектура и мноштво опционих екстензија. Зато је \рисцв{} архитектура одабрана за имплементацију на процесору који је тема овог рада.

Проблем којим се овај рад директно бави се тиче отклањања грешака у софтверу који се извршава на \рисцв{} процесорима.
Временом софтвер постаје све комплекснији а самим тим и подложнији грешкама. 
Један од најчешћих алата који се користе при проналажењу грешака је дебагер који омогућава контролу извршавања програма и увид у његово стање.
Уобичајено је да подршку за дебаговање пружа оперативни систем, међутим већина уграђених (енг. \textit{embedded}) система имају веома просте (или уопште немају) оперативне системе.
Код таквих система се подршка за дебаговање реализује директно у хардверу, тако што се обезбеђује посебан интерфејс преко којег се (уз помоћ посебног адаптера) циљни систем повезује на софтвер који управља дебаговањем а извршава се на десктоп рачунару.

Како је подршка за екстерно дебаговање неопходна за било коју озбиљну комерцијалну имплементацију, \рисцв{} организација је дефинисала спецификацију за екстерно дебаговање\cite{debug_spec}, коју је већина комерцијалних имплементација усвојила.\newpage

Циљ рада је имплементација \рисцв{} процесора са подршком за екстерно дебаговање која поштује званичну спецификацију\cite{debug_spec} ради бољег разумевања, проналажења потенцијалних унапређења и процене комплексности имплементирања исте у другим истраживачким пројектима.

Имплементација је реализована на \textit{\acrfull{fpga}} чипу, те су поред симулација вршени тестови и на конфигурисаном хардверу.
Имплементација се састоји од самог \рисцв{} процесорског језгра (у даљем тексту само језгро), меморије за програм и податке, модула који обезбеђује подршку за екстерно дебаговање (у даљем тексту \textit{\acrfull{DM}}), модула који омогућава повезивање са рачунаром који управља дебаговањем (у даљем тексту \textit{\acrfull{DTM}}) и неколико периферија које доприносе живописнијој демонстрацији система. Ове компоненте су повезане коришћењем три магистрале, прва повезује језгро, меморију и периферије, друга језгро и \дм{} и трећа \дм{} и \дтм{}.

У 2. поглављу је дат кратак опис \рисцв{} спецификације\cite{riscv_spec} и спецификације за екстерно дебаговање\cite{debug_spec}, поређење са једном комерцијалном имплементацијом, као и опис технологија и алата коришћених при изради рада.
Поглавље 3 садржи детаљнији преглед целог система и његових мање битних компоненти.
Поглавља 4, 5 и 6 се фокусирају на детаље имплементације језгра, \дм{}-а и \дтм{}-а, док се поглавље 7 односи на софтверску страну екстерног дебаговања и њену конфигурацију.
На крају се у поглављу 8 приказује методика и резултати тестирања.
