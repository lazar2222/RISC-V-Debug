\chapter{Преглед спецификација и коришћених технологија}

За читаоце који нису упућени у \рисцв{} екосистем, у овом поглављу је дат кратак преглед најбитнијих делова спецификације.
Приказано је и поређење имплементације дате у овом раду са \textit{Si Five Freedom E310-G002} процесором коришћеном на \textit{Si Five HiFive RevB} \cite{hifive_site} развојној плочи.
Тај процесор је изабран због своје популарности и зато што је дизајниран за примену у уграђеним системима и као такав је по перформансама и способностима најближи имплементираном процесору.
На крају је дат и кратак опис технологија и алата коришћених у изради рада.

\section{\рисцв{} инструкцијски сет}

Како би постигао примењивост у широком опсегу имплементација са различитим циљевима, \рисцв{} користи модуларан приступ, спецификација \cite{riscv_spec} прописује неколико основних инструкцијских сетова, као и велики број опционих екстензија.
Основни инструкцијски сет прописује око 40 обавезних инструкција које обухватају аритметичке и логичке инструкције, инструкције за приступ меморији и инструкције контроле тока.
Тренутно су ратификована два основна инструкцијска сета: \textit{RV32I} i \textit{RV64I}. Оба сета прописују 32 регистра опште намене а разликују се у ширини регистара (самим тиме и у величини меморијског простора) која је 32 и 64 бита респективно. Такође су предложена још два основна сета, један који смањује број регистара на 16 и један који повећава ширину регистара и меморијског простора на 128 бита.

Поред основног инструкцијског сета, \рисцв{} организација је ратификовала и 8 опционих екстензија које проширују способности процесора.

То су:
\begin{itemize}
	\item \textbf{M} - Целобројно множење, дељење и остатак при дељењу
	\item \textbf{A} - Атомичне операције над меморијом
	\item \textbf{F} - Операције над бројевима у покретном зарезу једноструке прецизности
	\item \textbf{D} - Операције над бројевима у покретном зарезу двоструке прецизности
	\item \textbf{Q} - Операције над бројевима у покретном зарезу четвороструке прецизности
	\item \textbf{C} - Компримоване (16 битне) инструкције
	\item \textbf{Zicsr} - Читање и писање контролних и статусних регистара
	\item \textbf{Zifencei} - Синхронизација уписа у програмску меморију
\end{itemize}

\рисцв{} такође предлаже и привилеговану архитектуру \cite{priv_spec} која садржи 3 нивоа извршавања: машински (\textit{M}), супервизорски (\textit{S}) и кориснички (\textit{U}).
Предложена привилегована архитектура такође подржава виртуализацију и садржи опис процедуре обраде прекида.

\section{Подршка за екстерно дебаговање}

Спецификација подршке за екстерно дебаговање \cite{debug_spec} се састоји из 4 дела чији су мањи или већи делови опциони.
То су: \дм{}, \дтм{}, посебан ниво извршавања у коме се процесор налази док је заустављен од стране дебагера (енг. \textit{debug (D)} мод) и хардверски окидачи.

\acrfull{DM} прима команде које долазе од софтвера који управља дебаговањем (које долазе преко \дтм{}-a i \textit{\acrfull{DMI}}-a) и на основу њих управља једним или више језгара.
Већина делова \дм{}-a су опциони али је неопходно да имплементирани делови омогућавају све потребне операције за успешно дебаговање, сама спецификација предлаже два подскупа спецификације који испуњавају овај услов.
\дм{} мора имплементирати контролу ресет сигнала језгра, механизам за покретање и заустављање језгра и приступ регистрима језгра.
\дм{} опционо може подржати приступ меморији из погледа језгра или коришћењем додатног газде на магистрали, извршавање произвољних инструкција и приступ контролним и статусним регистрима.

\acrfull{DTM} прима команде које долазе од софтвера који управља дебаговањем одабраним протоколом и преводи их у приступе \дми{} магистрали.
\дтм{} може користити било који протокол али у спецификацији постоји само опис \дтм{}-а који користи \textit{\acrfull{JTAG}} протокол.
Као такав, \дтм{} је неопходан али није неопходно да имплементирани \дтм{} користи \јтаг{} протокол.

Имплементација \дмоде{} мода је обавезна и састоји се од малих промена понашања језгра у односу на \textit{M} мод извршавања и неколико додатних контролних и статусних регистара доступних само \дм{}-у.

Хардверски окидачи су опциони али могу бити имплементирани и независно од остатка спецификације јер могу бити корисни и када је подршка за дебаговање имплементирана у софтверу.
Спецификација омогућава произвољан број хардверских окидача као и произвољан избор које функционалности окидача су имплементиране.
У основи постоје 4 типа окидача који унутар себе имају велики број функционалности које су опционе и у које се неће улазити сада.
\begin{itemize}
	\item Окидач на адресу или податак учитан из меморије (ово обухвата читање и упис у меморију као и дохватање и извршавање инструкције)
	\item Окидач на број извршених инструкција (ово је један од начина за имплементацију проласка кроз програм инструкцију по инструкцију)
	\item Окидач на обраду изузетка
	\item Окидач на обраду прекида
\end{itemize}

Када се окидач окине, у зависности од конфигурације, језгро може генерисати изузетак или прећи у \дмоде{} мод и стати са извршавањем.

\section{Поређење са \textit{Si Five Freedom E310-G002}}

На табели \ref{table:comp} се налази поређење имплементираног процесора и подршке за дебаговање са \textit{Si Five Freedom E310-G002} \cite{hifive_manual} по неким од параметара изнетих у претходне две секције.

\begin{table}[h!]
	\centering
	\caption{Поређење \textit{Si Five Freedom E310-G002} са имплементираним процесором}
	\label{table:comp}
	\begin{tabular}{|p{5.5cm}|l|l|} 
		\hline \rowcolor{lightgray} & \textit{Si Five Freedom E310-G002} & Имплементирани процесор \\
		\hline Основни инструкцијски сет & RV32I & RV32I \\
		\hline Подржане екстензије & M, A, C, Zicsr & Zicsr \\
		\hline Подржани модови извршавања & M, U & M \\
		\hline Организација & Проточна обрада са 5 корака & Вишециклична \\
		\hline Максимални \acrshort{ИПЦ}\footnotemark & 1 & 1 \\
		\hline Фреквенција сигнала такта & до 384 MHz & 35 MHz \\
		\hline Механизам за покретање, ресетовање и заустављање језгра & Да & Да \\
		\hline Приступ регистрима опште намене & Да & Да \\
		\hline Приступ контролним и статусним регистрима & Не & Да \\
		\hline Приступ регистрима без заустављања језгра & Не & Не \\
		\hline Извршавање произвољних инструкција & Да & Да \\
		\hline Величина бафера за произвољне инструкције & 16 меморијских речи & 16 меморијских речи \\
		\hline Приступ меморији из погледа језгра & Не & Да \\
		\hline Приступ меморији из погледа језгра без заустављања језгра & Не & Не \\
		\hline Број помоћних регистара\footnotemark & 1 + 1 & 2 + 12 \\
		\hline Приступ меморији коришћењем додатног газде на магистрали & Не & Да \\
		\hline Број хардверских окидача & 8 & 4 \\
		\hline Окидач на адресу или податак подржан & Да & Да\footnotemark \\
		\hline Окидач на број извршених инструкција подржан & Не & Не \\
		\hline Окидач на обраду изузетка подржан & Не & Не \\
		\hline Окидач на обраду прекида подржан & Не & Не \\
		\hline \дтм{} протокол & \јтаг{} & \јтаг{} \\
		\hline Препоручена фреквенција \јтаг{} интерфејса & 4MHz & 1MHz \\
		\hline Интегрисан \јтаг{} адаптер & Да (J-Link OB) & Не \\
		\hline
	\end{tabular}
\end{table}

\footnotetext[1]{\acrfull{ИПЦ}.}
\footnotetext[2]{Мапираних као контролни и статусни регистри + меморијски мапираних.}
\footnotetext[3]{Окидач не подржава комплетан сет опционих функционалности.}

Као што се може видети на табели \ref{table:comp}, \textit{E310-G002} је знатно софистициранији процесор, што је и разумљиво, јер је у питању комерцијална имплементација.
Међутим подршка за дебаговање на \textit{E310-G002} обухвата један од препоручених минималних подскупа функционалности. 
Имплементирани подскуп функционалности се базира на извршавању произвољних инструкција (функционалности које нису директно подржане се могу емулирати извршавањем произвољног кода који користи помоћне регистре).

Како је фокус рада примарно на подршци за дебаговање, нешто једноставнији процесор са опширнијом подршком за дебаговање представља логичан избор.

\section{Преглед коришћених технологија и алата}

\subsection{\фпга{}}

\textit{\acrfull{fpga}} су интегрисана кола чија се функционалност може мењати по потреби. За разлику од процесора који су такође програмабилни, \фпга{} не извршава код већ директно имплементира тражени дизајн на нивоу дигиталне логике. Дизајн за \фпга{} je репрезентован битским током (енг. \textit{bitstream}) који посебни алати за синтезу генеришу користећи дизајн написан у неком од језика за опис хардвера.

\фпга{} чипови су интерно реализовани коришћењем логичких елемената (енг. \textit{\acrfull{LE}}) који се састоје од лукап табеле (која може да репрезентује произвољну комбинациону логику) и Д флип-флопа.
Велики број логичких елемената (неколико десетина хиљада) је међусобно повезано конфигурабилним везама тако да се произвољни улази и излази логичких елемената могу повезати.
\фпга{} чипови поред ове конфигурабилне логике често имају и додатне компоненте које олакшавају имплементацију одређених решења, то су обично интегрисане меморије, сабирачи, множачи, фазно закључане петље (енг. \textit{\acrfull{PLL}}) итд.

За имплементацију процесора приказаног у овом раду коришћен је \textit{Altera Cyclone V \\ 5CSXFC6D6F31C6N} \cite{cycv_hb} \фпга{} чип.
\textit{5CSXFC6D6F31C6N} садржи 110 хиљада логичких елемената, 5761 килобита интегрисане меморије, 6 фазно закључаних петљи, 2 интегрисана меморијска контролера и двојезгарни \textit{ARM} микропроцесор (меморијски контролери и \textit{ARM} микропроцесор нису коришћени у овом раду). 

\subsection{\textit{System Verilog}}

\textit{System Verilog} \cite{sv_spec} је језик за опис хардвера, настао као проширење на \textit{Verilog}.
Језици за опис хардвера омогућавају дизајнерима хардвера да коришћењем програмског кода, на једноставан начин опишу структуру и понашање жељеног хардверског дизајна.
Дизајн описан у језику за опис хардвера се након креирања може симулирати, претворити у шаблон за производњу интегрисаних кола или битски ток за конфигурацију \фпга{} чипа.

Једна од предности \textit{System Verilog}-а у односу на \textit{Verilog} је постојање интерфејса који представљају именовану групу сигнала, што их чини идеалним за репрезентовање магистрала и представља примарни разлог његовог избора за имплементацију процесора у овом раду.

\subsection{\textit{Quartus} и \textit{Questa}}

За синтезу дизајна написаног у језику за опис хардвера коришћен је \textit{Intel Quartus Prime Lite} \cite{quartus_man} верзија \textit{22.1std.1}, док је за симулацију и прелиминарно тестирање дизајна коришћена \textit{Questa Intel Starter FPGA Edition-64} \cite{questa_man} верзија \textit{2012.2}.


\subsection{\textit{Eclipse Embedded CDT}}

\textit{Eclipse} је лако прошириво интегрисанo развојно окружење (енг. \textit{\acrfull{IDE}}) које подржава велики број програмских језика, а уз \textit{Embedded \acrfull{CDT}} \cite{embcdt} сет екстензија је специјализовано за развој уграђених система. \textit{Embedded \acrshort{CDT}} сет екстензија између осталог садржи подршку за \textit{Make} систем за превођење кода, као и директну подршку за дебаговање коришћењем \textit{OpenOCD}-a или \textit{J-Link} софтвера.
Из тих разлога се за сав развој и дебаговање софтвера који се извршава на имплементираном процесору користи \textit{Eclipse Embedded CDT} верзија \textit{2023-06}.

\subsection{\textit{GCC} и \textit{OpenOCD}}

\textit{\acrfull{GCC}} \cite{gcc} је добро познати пакет програмских преводилаца и пратећих алата, који обухвата и укрштени преводилац (енг. cross-compiler) за \рисцв{} архитектуру.
За превођење софтвера писаног у \textit{C} програмском језику за имплементирани процесор коришћена је верзија \textit{13.2.0 \acrshort{GCC}}-a која је део пакета \textit{xPack GNU RISC-V Embedded GCC x86\_64} који је препоручен од стране \textit{Eclipse Embedded CDT} развојног окружења.

\textit{Open \acrfull{OCD}} \cite{openocd} је пројекат отвореног кода који за циљ има да премости јаз између унификованог интерфејса који \textit{\acrfull{GDB}} пружа програмерима и интерфејса који пружа сам циљни процесор.
Ово је реализовано кроз три нивоа апстракције. На самом дну имамо ниво адаптера за дебаговање, који комуницира са адаптером за дебаговање и користећи њега извршава просте операције комуникационог протокола који циљни процесор користи. Изнад тога се налази ниво архитектуре процесора који операције, као што су читање регистара или заустављање процесора, преводи у секвенцу операција комуникационог протокола који се користи. На врху се налази ниво конкретног система где се архитектури процесора додају информације о броју језгара, меморијској мапи и повезаним периферијама.

Како \textit{Open\acrshort{OCD}} већ има уграђену подршку за верзију спецификације \cite{debug_spec} коришћену у овом раду, као и \textit{J-Link} (коришћени адаптер за дебаговање), потребно је само дефинисати конфигурациони фајл за ниво система. У раду је коришћена верзија \textit{0.12.0}, такође део пакета \textit{xPack OpenOCD} препорученог од стране \textit{Eclipse Embedded CDT}.

\subsection{\textit{\acrshort{JTAG}}}

\textit{\acrfull{JTAG}} \cite{jtag_spec} je комуникациони протокол, оригинално дизајниран за тестирање интегрисаних кола и штампаних плоча, који је убрзо усвојен и од стране произвођача микропроцесора као протокол за комуникацију са подршком за екстерно дебаговање. Протокол је детаљније објашњен у поглављу 6 које се бави имплементацијом \дтм{}-а.

\subsection{\textit{J-Link}}

\textit{J-Link} \cite{jlink} је један од најпопуларнијих адаптера за дебаговање који подржава \textit{\acrshort{JTAG}} протокол.
\textit{J-Link} се може у овом случају користити на два начина, директно са пратећим софтвером или коришћењем \textit{Open\acrshort{OCD}}-а. Уколико се користи са пратећим софтвером, тај софтвер игра сличну улогу као \textit{Open\acrshort{OCD}} али због бољег познавања хардвера може ефикасније да га искористи што резултује у бољем времену одзива при дебаговању. Уколико се користи \textit{Open\acrshort{OCD}}, \textit{J-Link} се понаша као прост адаптер и \textit{Open\acrshort{OCD}} мора да управља њиме на нижем нивоу. Иако су предности првог приступа очигледне, у духу пројекта отвореног кода у поглављу 7 су приказана оба метода.