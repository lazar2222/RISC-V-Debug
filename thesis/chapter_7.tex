\chapter{Резултати}

У овом поглављу ће укратко бити објашњено које су методе тестирања коришћене при изради рада и њихови резултати.

При имплементацији језгра, за тестирање су коришћени кодови и периферије оригинално написане за \cite{arilla}. Кодови су покретани и унутар симулатора и на \textit{\acrshort{fpga}} чипу а успешност тестова је ручно процењивана.
При имплементацији екстензија језгра и подршке за дебаговање је за тестирање примарно коришћена симулација са ручном манипулацијом сигнала и проценом успешности тестова.
Када је подршка за екстерно дебаговање комплетно имплементирана први корак тестирања је био провера правилне детекције свих параметара и способности језгра. То је вршено коришћењем командне линије \textit{J-Link} софтвера.

\lstinputlisting[language=none,caption=Пример успешне детекције способности језгра]{listing/det.txt}

Након конфигурације интегрисаног развојног окружења и писања тест кода који коришћењем тајмера одбројава секунде и приказује их на седмосегментним дисплејима, мануелно тестирање функционалности подршке за екстерно дебаговање је вршено и коришћењем командне линије \textit{J-Link} софтвера, као и коришћењем интегрисаног развојног окружења.
Одређене функционалности подложне грешкама (апстрактне команде и приступ меморији) су тестиране у симулатору. За потребе овог тестирања симулатор је аутоматизован скриптама које аутоматизују побуде, али је процена успешности тестова и даље остала мануелна. Скрипте коришћене за ове тестове су приложене у додатку \ref{chap:scr}.
Кроз све фазе тестирања дизајн је синтетисан и вршена је мануелна инспекција генерисане нетлисте и порука пријављених при синтези. Такође је увек вршена аутоматска провера временских ограничења.

На самом крају је комплетан хардвер тестиран једним сетом тестова који је саставила \рисцв{} организација, овај сет тестова тестира језгро и подршку за екстерно дебаговање заједно.
Тест је реализован покретањем \textit{\acrshort{GDB}}-a који се преко \textit{Open\acrshort{OCD}}-а и адаптера повезује на језгро које се извршава на \textit{\acrshort{fpga}} чипу и задавањем предефинисаних команди и поређењем излаза. Конфигурација потребна за овај вид тестирања се састоји од \textit{Open\acrshort{OCD}} конфигурационог фајла, скрипте за повезивач и пајтон фајла који садржи податке о језгру. Ови фајлови су такође приложени у додатку \ref{chap:conf}. Тестови који су ручно означени као неприменљиви су тестови у којима аутоматска детекција није правилно детектовала да нека функционалност није подржана, што се може сматрати грешком у тесту. Треба напоменути да су ови тестови застарели али да актуелни сет тестова не садржи тестове за подршку за екстерно дебаговање и да је примарно дизајниран за тестирање имплементација у симулатору. Тестирање језгра овим тестовима је разматрано, међутим повезивање имплементације са тим тестовима је значајно компликованије и ван обима овог рада.

Резултат извршавања ових тестова је \textbf{27 неприменљивих} тестова и \textbf{45 успешних} тестова, од укупно \textbf{72 извршена} теста.
Детаљнији запис извршавања тестова је доступан у  додатку \ref{chap:sum}.

Треба напоменути да ништа од горе наведених тестова није замена са праву и темељну верификацију система, али је тако нешто далеко ван обима овог рада и може представљати рад сам за себе.

