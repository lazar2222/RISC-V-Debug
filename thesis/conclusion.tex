\chapter{Закључак}

Циљ овог рада је било имплементирање \рисцв{} језгра и подршке за његово екстерно дебаговање која поштује званичну спецификацију \cite{debug_spec}, ради бољег разумевања саме спецификације и потенцијалних унапређења. Од велике важности је такође било демонстрирати функционисање овог система са стандардним алатима које програмери заправо користе за проналажење и отклањање грешака у коду.

На основу тестирања приказаног у прошлом поглављу је могуђе закључити да је имплементациони део рада успешно реализован, и да је процес имплементације резултовао у усвајању дубоког знања које се тиче имплементираних спецификација.

Уочено је да је фокус на перформансама језгра (иако је резултовао у бољим перформансама у односу на претходну имплементацију) значајно повећао комплексност решења, те се за озбиљније имплементације предлаже одабир организације са проточном обрадом, или неке још модерније организације.
Имплементација комплетне спецификације подршке за екстерно дебаговање је прилично комплексна и временски захтевна, без великог бенефита у подржаној функционалности. Ово је наравно одлика спецификације и може се рећи да се и не очекује имплементација комплетне спецификације већ избор једног од могућих подскупа који пружају све потребне функционалности. Како је циљ рада био разумевање комплетне спецификације имплементација великог дела спецификације се сматра оправданом, међутим за будуће имплементације се топло препоручује избор једно од могућих подскупа. Такође када се гледају ресурси потребни за имплементацију минималног подскупа и вредност коју подршка за екстерно дебаговање доноси, сматра се да је имплементација исте оправдана или чак препоручена. Још једна опсервација је да управо тај приступ произвољног избора подскупа који чини спецификацију повољном за имплементацију, представља највећу препреку постизању добре софтверске подршке. Софтвери ће се често фокусирати на најчешће имплементиране подскупе са слабом или непостојећом подршком за остале функционалности. Ово доводи до тога да софтвери не искоришћавају неке комплексне функционалности иако су доступне у хардверу јер нису довољно честе. Што умањује подстицај дизајнерима хардвера да те функционалности имплементирају, на крају резултујући у конвергенцији на неколико минималних подскупа са мнималним функционалностима неопходним за дебаговање.

Даљи рад у овој области се може фокусирати на имплементирано језгро и проширивање његове применљивости имплементацијом додатних екстензија инструкцијског сета или побољшање његових перформанси коришћењем алтернативних организација језгра.
На пољу саме подршке за дебаговање се може радити на имплементацији додатних опционих функционалности или других типова хардверских окидача, као и подршци за манипулацију регистрима без заустављања језгра.
Међутим вероватно најплоднија област за даље истраживање се тиче неинвазивног праћења рада система тј. трагова извршавања. Тренутно постоје две предложене спецификације за трагове извршавања на \рисцв{} системима. Те се детаљна анализа њихових предности и мана, као и предлог начина за постизање паритета функционалности између њих се сматра преко потребним.